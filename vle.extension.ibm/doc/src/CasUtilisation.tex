\documentclass[a4paper,11pt,final]{article}
\usepackage[utf8]{inputenc} % Prendre en compte les caractères accentués
\usepackage[francais]{babel} % Prendre en compte les particularités de la typographie française.
\usepackage{geometry}         % marges
\usepackage{graphicx}         % images
\usepackage{setspace}
\usepackage[french]{varioref}
\usepackage{titlesec}
\usepackage{parskip}
\usepackage{url}
\usepackage{verbatim}
\usepackage{caption}
\usepackage{enumitem}
\usepackage{ragged2e}
\usepackage{xcolor}
%\usepackage{listing}
%\usepackage[T1]{fontenc}
%\usepackage{inconsolata}
\usepackage{textcomp}
\usepackage{listings}
%\usepackage{enumerate}


\lstdefinestyle{myLuastyle}
{
  language         = {[5.0]Lua},
  basicstyle       = \ttfamily,
  showstringspaces = false,
  upquote          = true,
  commentstyle     =\color{gray},
  stringstyle=\color{blue}
}

\lstset{style=myLuastyle}

\lstset{emph={%  
    ibm, addModel, getNumber, getModelValue, setModelValue, getName, addEvent, addModelWithParam, delModel, getTime, getModelName%
    },emphstyle={\color{red}\bfseries}%
}

\titlespacing{\section}{0pt}{*2}{*2}
\titleformat{\section}[hang]{\bf\large}{Cas \thesection ~:}{0.5pc}{}
\renewcommand{\baselinestretch}{1.2}
\setlength{\parskip}{1.5ex plus .4ex minus .4ex}
\setlength{\parindent}{15pt} 
\setlength{\topmargin}{-35pt}
\setlength{\textheight}{650pt}
\renewcommand{\theenumii}{\arabic{enumi}.\arabic{enumii}}

\title{\textbf{Cas d'utilisations plugin IBM}}
\author{}
\date{}
\begin{document}

\maketitle

\section{Créer un nouveau projet et vpz, simuler la création d'un individu}

\begin{enumerate}
  \item Créer un nouveau projet
  \item Créer un nouveau Vpz test\_0
  \item Ouvrir test\_0.vpz avec le plugin IBM
  \item Créer une nouvelle classe Loup avec un compartiment C1 = 1
  \begin{enumerate}
  	\item Clique-droit sur la colonne des classes, add.
  	\item Entrer le nom ``Loup''
  	\item Le plugin Forrester s'ouvre, définir la dynamique de l'individu ici
  	\item Valider
  \end{enumerate}
  \item Ajouter au Script principal
\begin{lstlisting}[frame=single]
ibm:addModel(1,"Loup")
\end{lstlisting}
  \item Valider, le Controleur se crée
  \item Mettre en place un observable avec une vue
  \begin{enumerate}
	  \item Créer une vue test\_0View, type file. Valider
	  \item Ajouter la vue au compartiment C1 de Loup : Aller dans Atomic Model Loup, éditer obs\_DTE\_Loup
	  \item Cocher le bouton radio, valider
	  \item Enregistrer
  \end{enumerate}
  \item Lancer la simulation /!\ Il manque vle.extension.differential-equation dans les Build-Depends de Description.txt
\end{enumerate}

\section{Simuler la création de plusieurs individus}
\begin{enumerate}
	  \item Ouvrir le test\_0.vpz avec le plugin IBM
	  \item Modifier la commande ibm\string:addModel(1,"Loup") pour créer 10 individus
	  \begin{lstlisting}[frame=single]
ibm:addModel(10,"Loup")
\end{lstlisting}
	  \item Lancer la simulation
\end{enumerate}

\section{Création de plusieurs individus grâce aux conditions expérimentales} 
\begin{enumerate}
	  \item Ouvrir les conditions expérimentales
	  \item Ajouter un paramètre ``nb\_Loup'' avec pour valeur 5.
	  \item Ouvrir ``test\_0.vpz'' avec le plugin
	  \item Mettre dans le script
	  \begin{lstlisting}[frame=single]
ibm:addModel("nb_Loup","Loup")
\end{lstlisting}
	  \item Simuler, remarquer 5 loups créés.
\end{enumerate}

\section{Création de plusieurs individus grâce aux conditions expérimentales} 
\begin{enumerate}
	  \item Ouvrir test\_0.vpz avec le plugin IBM
	  \item Mettre dans le script
	  \begin{lstlisting}[frame=single]
ibm:addModelWithParam(3,"Loup", "C1", 15)
\end{lstlisting}
	  \item Simuler, remarquer le changement de valeur de C1
\end{enumerate}

\section{Plusieurs individus, modification d'un paramètre grâce aux conditions expérimentales} 
\begin{enumerate}
	  \item Ouvrir les conditions expérimentales
	  \item Ajouter le paramètre ``dix'' = 10 à cond\_Controleur
	  \item Ouvrir test\_0.vpz avec le plugin IBM
	  \item Mettre dans le script 
	  \begin{lstlisting}[frame=single]
ibm:addModelWithParam(3,"Loup", "C1", "dix")
\end{lstlisting}
	\item Simuler, remarquer le changement de valeur de C1
\end{enumerate}

\section{Plusieurs individus avec des paramètres distincts} 
\begin{enumerate}
	  \item Ouvrir le test\_0.vpz avec le plugin IBM
	  \item Mettre dans le script 
	  \begin{lstlisting}[frame=single]
for i=1,5,1 do
	ibm:addModelWithParam(1,"Loup", "C1", i)
end
\end{lstlisting}
	\item Simuler, remarquer le changement de valeur de C1
\end{enumerate}

\section{Plusieurs individus, modification de plusieurs paramètres} 
\begin{enumerate}
	  \item Ouvrir test\_0.vpz avec le plugin IBM
	\item Sélectionner Loup
	\item Faire un clique droit et selectionner Modify
	\item Ajouter un compartiment C2 = 0
	\item Mettre dans le script 
\begin{lstlisting}[frame=single]
for i=1,5,1 do
	ibm:addModelWithParam(1,"Loup", "C1", i, "C2", 3)
end
\end{lstlisting}
\item Ajouter un observable de C2
\begin{enumerate}
	  \item Ouvrir la fenêtre de dialogue Atomic Model de Loup
	  \item Aller dans l'onglet Observable et Editer obs\_DTE\_Loup
	  \item Ajouter une vue à C2
\end{enumerate}
\item Simuler, remarquer le changement de valeur de C1 et de C2
\end{enumerate}

\section{Créer un évènement d'initialisation}
\begin{enumerate}
	\item Ouvrir test\_0.vpz avec le plugin IBM
	\item Ajouter un script grâce au bouton ``+'', l'appeler ``init'', valider
	\item Une nouvelle ligne s'affiche avec le nom ``init'', cliquer sur la flèche, une zone de texte apparait, c'est là que l'évènement ``init'' doit être décrit.
	\item Couper-coller le texte du Script principal dans le script init.
	\item Mettre dans le Script principal
	\begin{lstlisting}[frame=single]
ibm:addEvent("INIT", "init")
\end{lstlisting} 
	\item Simuler, remarquer le non changement depuis le cas précédent
\end{enumerate}

\section{Modifier les scripts depuis les conditions expériementales}
\begin{enumerate}
	\item Ouvrir les conditions expérimentales
	\item Dans cond\_Controleur remarquer les paramètres Script et init qui correspondent au script principal et à l'évènement ``init'' créés.
	\item Modifier init par
	\begin{lstlisting}[frame=single]
for i=1,5,1 do
	ibm:addModelWithParam(1,"Loup", "C1", i)
end
	\end{lstlisting}  

	Seuls les paramètres C1 seront modifiés
	\item Valider
	\item Simuler, remarquer que tous les C2 ont la valeur par défaut 0
\end{enumerate}

\section{Ajouter un évènement SINGLE}
\begin{enumerate}
	\item Ouvrir test\_0.vpz avec le plugin IBM
	\item Ajouter un nouveau Script. L'appeler ``single''
	\item Dans la zone de script de "single" mettre 
	\begin{lstlisting}[frame=single]
print("Il est : ", ibm:getTime());
	\end{lstlisting} 
	\item Dans la zone de Script principal mettre 
	\begin{lstlisting}[frame=single]
ibm:addEvent("SINGLE",50,"single");
	\end{lstlisting}
	\item Simuler, remarquer l'affichage de "Il est : 50" dans la console à la date 50.
\end{enumerate}

\section{Envoyer une valeur à un individu}
\begin{enumerate}
\item Ouvrir le test\_0.vpz avec le plugin IBM
\item Dans le script de ``single'' créé précédemment ajouter 
\begin{lstlisting}[frame=single]
ibm:setModelValue("Loup", 3, "C1", 0)
	\end{lstlisting} 
\item Simuler, remarquer le changement de la valeur C1 du 3ème Loup à partir de la date 50.
\end{enumerate}

\section{Envoyer une valeur des conditions expérimentales à un modèle}
\begin{enumerate}
\item Ouvrir le test\_0.vpz avec le plugin IBM
\item Dans le script de ``single'' créé précédemment ajouter
\begin{lstlisting}[frame=single]
ibm:setModelValue("Loup", 3, "C1", "dix")
	\end{lstlisting} 
\item Simuler, remarquer le changement de la valeur C1 du 3ème Loup à partir de la date 50.
\end{enumerate}

\section{Envoyer une valeur à un individu grâce à son nom}
\begin{enumerate}
\item Ouvrir le test\_0.vpz avec le plugin IBM
\item Dans le script de ``single'' créé précédemment ajouter 
\begin{lstlisting}[frame=single]
ibm:setModelValue("Loup_0", "C1", "dix")
	\end{lstlisting} 
\item Simuler, remarquer le changement de la valeur C1 de ``Loup\_0'' à partir de la date 50.
\end{enumerate}

\section{Obtenir la valeur d'un paramètre d'un individu et appel à plusieurs évènements SINGLE}
\begin{enumerate}
\item Ouvrir le test\_0.vpz avec le plugin IBM
\item Créer un nouvel évènement SINGLE appelé ``single2''
\item Dans le script de ``single2'' nouvellement créé, ajouter
\begin{lstlisting}[frame=single]
print("valeur de C1:", ibm:getModelValue("Loup", 1,"C1"))
	\end{lstlisting} 
\item Modifier le Script principal par
\begin{lstlisting}[frame=single]
ibm:addEvent("INIT","init")
ibm:addEvent("SINGLE",49,"single2");
ibm:addEvent("SINGLE",50,"single");
ibm:addEvent("SINGLE",51,"single2");
	\end{lstlisting} 
\item Simuler, remarquer la trace dans la console. /!\ Modifier une valeur et lui demander directement après sa valeur ne fonctionne pas.
\end{enumerate}

\section{Obtenir la valeur d'un paramètre d'un individu et appel à plusieurs évènements SINGLE à partir du nom d'un individu}
\begin{enumerate}
\item Reprendre le cas 14 et  
\item Mettre dans le script de "single2"
\begin{lstlisting}[frame=single]
print("valeur de C1 :", ibm:getModelValue("Loup_0","C1"))
	\end{lstlisting} 
\item Simuler, remarquer le changement de la valeur C1 de ``Loup\_0'' à partir de la date 50.
\end{enumerate}

\section{Obtenir le nom d'un individu}
\begin{enumerate}
\item Ouvrir le test\_0.vpz avec le plugin IBM 
\item Nettoyer les évènements SINGLE des cas précédents, suppression de single2 (selectionner la ligne de single2 et cliquer sur ``-'') et des lignes ibm\string:addEvent(``SINGLE'',49,``single2''); et ibm\string:addEvent(``SINGLE'',51,``single2'');
\item Mettre dans single
\begin{lstlisting}[frame=single]
print(ibm:getModelName("Loup", 4))
	\end{lstlisting} 
\item Simuler, remarquer la trace dans la console à la date 50, c'est le nom du 4ème Loup.
\end{enumerate}

\section{Obtenir le nom de tous les individus}
\begin{enumerate}
	\item Ouvrir le test\_0.vpz avec le plugin IBM 
	\item Mettre dans single le script qui itère tous les individus Loup, ibm\string:getNumber renvoie le nombre de loup vivants
	\begin{lstlisting}[frame=single]
for i=1,ibm:getNumber("Loup"), 1 do 
	print(ibm:getModelName("Loup", i))
end
		\end{lstlisting} 
	\item Simuler, remarquer la trace dans la console à la date 50, nom de tous les loups vivants.
\end{enumerate}

\section{Supprimer un individu}
\begin{enumerate}
	\item Ouvrir le test\_0.vpz avec le plugin IBM 
	\item Mettre dans single
	\begin{lstlisting}[frame=single]
ibm:delModel("Loup_0")
		\end{lstlisting} 
	\item Simuler, remarquer la mort de l'individu nommé ``Loup\_0'' dans les observables
\end{enumerate}

\section{Évènements réguliers}
\begin{enumerate}
	\item Ouvrir le test\_0.vpz avec le plugin IBM 
	\item Mettre dans single
	\begin{lstlisting}[frame=single]
print(ibm:getTime())
		\end{lstlisting} 
	\item Mettre dans le script principal
	\begin{lstlisting}[frame=single]
ibm:addEvent("INIT","init")
ibm:addEvent("RECUR",10,2,"single");
		\end{lstlisting}
	\item Simuler, remarquer la trace dans la console
\end{enumerate}

\section{Observer une variable lua}
\begin{enumerate}
	\item Ouvrir le test\_0.vpz avec le plugin IBM 
	\item Mettre dans single
	\begin{lstlisting}[frame=single]
acc = 0
for i=1,ibm:getNumber("Loup"),1 do
	acc = acc + ibm:getModelValue("Loup",i,"C1");
end
		\end{lstlisting} 
	\item Mettre dans le script principal
	\begin{lstlisting}[frame=single]
ibm:addEvent("INIT","init")
ibm:addEvent("RECUR",10,1,"single");
		\end{lstlisting} 
	\item Laisser init du Cas 9. Valider.
	\item Ouvrir Atomic Model du Controleur
	\item Aller dans l'onglet Observables et ajouter obs\_Controleur
	\item Editer obs\_Controleur
	\begin{enumerate}
		\item Clique-droit dans la colonne Observable, add
		\item Entrer le nom ``acc'' de notre nom de variable. Valider.
		\item Ajouter la vue test\_0View à acc. Valider.
	\end{enumerate}
	\item Cocher le bouton radio de obs\_Controleur
	\item Enregister, simuler
\end{enumerate}

\section{Se servir d'un évènement comme fonction}
\begin{enumerate}
	\item Ouvrir le test\_0.vpz avec le plugin IBM 
	\item Diminuer le nombre d'individus créés dans init à 2 ou 1
	\item Mettre dans le single init (), ce qui appellera la fonction init
	\item Simuler
\end{enumerate}
\end{document}
