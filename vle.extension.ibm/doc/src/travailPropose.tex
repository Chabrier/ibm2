\chapter{Description du travail proposé}
\setlength{\parskip}{2.5ex plus .4ex minus .4ex}

\section{Besoin}
Actuellement, le logiciel VLE permet de créer des individus un à un leur associer une dynamique, de les relier entre eux et de lancer des simulations. La dynamique de plusieurs individus peut être la même, et peut varier selon les conditions expérimentales qui décrivent les valeurs des paramètres si l'utilisateur les édite.\\
Cependant, il est nécessaire que l'utilisateur édite chaque individu un à un. Comment faire pour modéliser des troupeaux entiers sans avoir ce travail répétitif ?\\
C'est pourquoi, il était nécessaire de créer une extension du logiciel pour ces cas là. Une extension qui permettrait la création, suppression et prise de décision sur les individus de manière simple et instinctive pour l'utilisateur.\\
L'utilisateur n'aurait donc pas besoin de programmer en C++ ni d'éditer des graphes ??? et réutiliserait le plugin Forrester pour définir les classes d'individus qui serveront de base pour créer chaque individu lors de la simulation.\\
\\
Deux tp m'ont été donné afin de me donner une idée des fonctionnalités attendues et de diriger la conception et le développement de mon plugin. Ces exemples ont été réalisés avec le logiciel Modelmaker. Il me fallait alors adapter les fonctionnalités proposées par Modelmaker afin que ces tp soient réalisables avec VLE.\\
\\
Le premier tp avait pour but de créer 5 modèles dont chacun d'eux possédait un compartiment, le premier se vidant dans le deuxième et si le deuxième atteingnait une certaine valeur, se vidait immédiatement dans un troisième compartiment. Chaque modèle ayant des paramètres de vidange différents.\\
D'autres part, il devait être possible d'offrir à l'utilisateur la possibilité d'observer une ou plusieurs valeurs en particulier. Dans ce cas précis, la somme de tous les compartiments qui se vident.\\
\\
Le second tp était plus développé. Il s'agissait de faire naitre 10 cellules à t=10, faire croître leur poids suivant des paramètres distints puis à t=30, repérer la plus grosse cellule et faire décroître les 9 autres.\\ Lorsque la cellule la plus grosse atteint 0.99, 10 autres naissances sont lancées. Si une cellule a un poids inférieur à 0.1, la tuer.

\section{IBM}
L'objectif est de pouvoir modéliser des individus qui ont chacun une dynamique qui peut être la même ou pas afin de lancer des simulations sur une durée déterminée par l'utilisateur.\\
Dans VLE, chaque individu est représenté par une ou plusieurs ``boite'' appelé modèle, associée chacune à une dynamique et il est possible d'étudier le comportement de cet individu lors de la simulation. C'est ce que nous appelons l'individu-centré.\\
Dans le cadre du stage, nous souhaitons démultiplier ces individus afin de pouvoir réaliser des simulations à l'échelle de groupe, de troupeau par exemple, en gardant une indépendance entre chaque individu. Nous parlerons donc ici d'IBM, Individual Based Model.

\subsection{IBM Conceptuel}
Chaque individu est un système d'équations différentielles défini par une classe C++. Cette classe est appelée classe d'individu. C'est elle qui doit être défini par l'utilisateur puis qui servira de base pour créer tous les individus de ce type là. C'est pourquoi chaque individu de même type, a la même dynamique, seuls les paramètres peuvent diverger entre eux si l'utilisateur souhaite en modifier un ou plusieurs.\\
Comme dit précédemment, chaque individu est indépendant, ils n'intéragissent pas directement les un avec les autres. Ils doivent communiquer par l'intermédiaire d'un même controleur auquels ils sont tous connectés.\\
Chaque port de sortie de chaque individu est relié aux ports d'entrée du controleur, ce qui permet à ce dernier de recevoir des évènements, les traiter et envoyer des réponses adéquates par l'intermédiaire de ses ports de sorties reliés à chaque individu.\\
Lors de la simulation, la dynamique du controleur se met en marche. Elle initialise tous les individus, exécute sa dynamique interne, reçoit les évènements externes et envoie des réponses aux individus. Durant la simulation, le controleur peut à tout moment, créer un nouvel individu, en supprimer ou en modifier les paramètres grâces aux évènements qu'il reçoit en entrée et qu'il peut envoyer en sortie.\\
Par exemple :\\
\begin{minipage}{\linewidth}% to keep image and caption on one page
\makebox[\linewidth]{%        to center the image
  \includegraphics [width=150mm]{images/exemple_ibm.png}}
\captionof{figure}{Trois Loups créés par le controleur lors de la simulation}%\label{visina8}%      only if needed  
\end{minipage}

Le controleur crée les individus ``Loup 1'', ``Loup 2'' et ``Loup 3'' qui sont bien indépendants les uns des autres et tous reliés au controleur comme expliqué au paragraphe précédent. \\

\subsection{IBM dans VLE}
Dans VLE, chaque système est représenté par un Vpz qui est un fichier xml où est décrit tout le système. Modèles présents (individus), conditions expérimentales (valeurs des paramètres), classes d'individus, ports d'entrées et de sorties...\\
Dans ce Vpz, un controleur est obligatoirement présent, il se crée automatiquement lors de la première ouverture du plugin IBM. \\Ensuite, les modèles souhaités par l'utilisateur sont créés lors de la simulation par le controleur grâce aux classes d'individu définies dans le vpz.\\
Par exemple : \\
\begin{minipage}{\linewidth}% to keep image and caption on one page
\makebox[\linewidth]{%        to center the image
  \includegraphics [width=150mm]{images/vpz1.png}}
\captionof{figure}{Composition d'un Vpz}%\label{visina8}%      only if needed  
\end{minipage}

Ici, le controleur crée n Loup et m Mouton lors de la simulation, mais seul le modèle ``Controleur'' est présent dans le vpz. Le controleur est cependant lié aux classes ``Loup'' et ``Mouton'' afin de pouvoir créer les individus.

\section{Fonctionnalités du plugin}
À son ouverture, le plugin récupère toutes les classes d'individu déjà présente dans le Vpz. Il offre ensuite la possibilité d'ouvrir le plugin Forrester afin de créer d'autres classes et modifier ou supprimer les classes existantes.\\
Afin de manipuler les individus lors de la simulation, le plugin propose un champs de texte afin que l'utilisateur exprime ces besoins par l'intermédiare d'un petit langage simple appelé Lua et quelques extensions que j'aurai développées.\\
Ces besoins peuvent être multiple, ils peuvent avoir un effet sur les individus, création, suppression, modification ou bien renvoyer des informations, valeur d'une variable, nombre d'individus, identifiant d'un individu...\\
D'autre part, le plugin crée automatiquement un exécutive appelé ``Controleur'' à son ouverture. Controleur qui sera chargé de manipuler les individus selon le script qu'aura écrit l'utilisateur, et la dynamique DEVS.\\