\chapter{Annexe}
\setlength{\parskip}{2.5ex plus .4ex minus .4ex}
\section{Présentation de l'entreprise}
\subsection{En France}
L'Institut national de la recherche agronomique (INRA) est un organisme français de recherche en agronomie fondé en 1946.\\
Elle produit des connaissances fondamentales afin d'apporter des innovations et des savoir-faire à la société dans divers domaines, changements climatiques, nutrition, biodiversité, carbone renouvelable...\\
Pour cela, elle est composée de 1 828 chercheurs dans le domaine des sciences du vivant, des sciences de la matière et des sciences humaines, 2 427 ingénieurs, 4 249 techniciens et administratifs répartis dans ??? sites dans toute la France. D'autre part, 1 784 doctorants et 1 000 stagiaires et chercheurs étrangers sont accueillis chaque année par l'institut.

\subsection{À Toulouse}
Présente depuis la fin du XIXème siècle, l'INRA de Toulouse accueille plus de 1600 personnes aujourd'hui, répartis en de nombreuses unités de recherche, expérimentales, plateformes...\\
Parmis ces dernières, il y a l'unité du MIAT (Mathématiques et Informatique Appliquées de Toulouse) qui est elle-même composée de 3 plateformes dont la plateforme RECORD (Rénovation et CooRDination de la modélisation de cultures pour la gestion des agro écosystèmes).

\subsection{L'équipe-projet RECORD}
RECORD est une plateforme de modélisation et de simulation des agro-écosystèmes. Elle propose un ensemble d'outils logiciels VLE, RVLE, pyVLE...\\
L'équipe-projet RECORD a donc pour but de